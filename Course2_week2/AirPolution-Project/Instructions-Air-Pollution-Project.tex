% Options for packages loaded elsewhere
\PassOptionsToPackage{unicode}{hyperref}
\PassOptionsToPackage{hyphens}{url}
%
\documentclass[
]{article}
\usepackage{lmodern}
\usepackage{amssymb,amsmath}
\usepackage{ifxetex,ifluatex}
\ifnum 0\ifxetex 1\fi\ifluatex 1\fi=0 % if pdftex
  \usepackage[T1]{fontenc}
  \usepackage[utf8]{inputenc}
  \usepackage{textcomp} % provide euro and other symbols
\else % if luatex or xetex
  \usepackage{unicode-math}
  \defaultfontfeatures{Scale=MatchLowercase}
  \defaultfontfeatures[\rmfamily]{Ligatures=TeX,Scale=1}
\fi
% Use upquote if available, for straight quotes in verbatim environments
\IfFileExists{upquote.sty}{\usepackage{upquote}}{}
\IfFileExists{microtype.sty}{% use microtype if available
  \usepackage[]{microtype}
  \UseMicrotypeSet[protrusion]{basicmath} % disable protrusion for tt fonts
}{}
\makeatletter
\@ifundefined{KOMAClassName}{% if non-KOMA class
  \IfFileExists{parskip.sty}{%
    \usepackage{parskip}
  }{% else
    \setlength{\parindent}{0pt}
    \setlength{\parskip}{6pt plus 2pt minus 1pt}}
}{% if KOMA class
  \KOMAoptions{parskip=half}}
\makeatother
\usepackage{xcolor}
\IfFileExists{xurl.sty}{\usepackage{xurl}}{} % add URL line breaks if available
\IfFileExists{bookmark.sty}{\usepackage{bookmark}}{\usepackage{hyperref}}
\hypersetup{
  pdftitle={Introduction Air Pollution Project},
  pdfauthor={R Programming},
  hidelinks,
  pdfcreator={LaTeX via pandoc}}
\urlstyle{same} % disable monospaced font for URLs
\usepackage[margin=1in]{geometry}
\usepackage{color}
\usepackage{fancyvrb}
\newcommand{\VerbBar}{|}
\newcommand{\VERB}{\Verb[commandchars=\\\{\}]}
\DefineVerbatimEnvironment{Highlighting}{Verbatim}{commandchars=\\\{\}}
% Add ',fontsize=\small' for more characters per line
\usepackage{framed}
\definecolor{shadecolor}{RGB}{248,248,248}
\newenvironment{Shaded}{\begin{snugshade}}{\end{snugshade}}
\newcommand{\AlertTok}[1]{\textcolor[rgb]{0.94,0.16,0.16}{#1}}
\newcommand{\AnnotationTok}[1]{\textcolor[rgb]{0.56,0.35,0.01}{\textbf{\textit{#1}}}}
\newcommand{\AttributeTok}[1]{\textcolor[rgb]{0.77,0.63,0.00}{#1}}
\newcommand{\BaseNTok}[1]{\textcolor[rgb]{0.00,0.00,0.81}{#1}}
\newcommand{\BuiltInTok}[1]{#1}
\newcommand{\CharTok}[1]{\textcolor[rgb]{0.31,0.60,0.02}{#1}}
\newcommand{\CommentTok}[1]{\textcolor[rgb]{0.56,0.35,0.01}{\textit{#1}}}
\newcommand{\CommentVarTok}[1]{\textcolor[rgb]{0.56,0.35,0.01}{\textbf{\textit{#1}}}}
\newcommand{\ConstantTok}[1]{\textcolor[rgb]{0.00,0.00,0.00}{#1}}
\newcommand{\ControlFlowTok}[1]{\textcolor[rgb]{0.13,0.29,0.53}{\textbf{#1}}}
\newcommand{\DataTypeTok}[1]{\textcolor[rgb]{0.13,0.29,0.53}{#1}}
\newcommand{\DecValTok}[1]{\textcolor[rgb]{0.00,0.00,0.81}{#1}}
\newcommand{\DocumentationTok}[1]{\textcolor[rgb]{0.56,0.35,0.01}{\textbf{\textit{#1}}}}
\newcommand{\ErrorTok}[1]{\textcolor[rgb]{0.64,0.00,0.00}{\textbf{#1}}}
\newcommand{\ExtensionTok}[1]{#1}
\newcommand{\FloatTok}[1]{\textcolor[rgb]{0.00,0.00,0.81}{#1}}
\newcommand{\FunctionTok}[1]{\textcolor[rgb]{0.00,0.00,0.00}{#1}}
\newcommand{\ImportTok}[1]{#1}
\newcommand{\InformationTok}[1]{\textcolor[rgb]{0.56,0.35,0.01}{\textbf{\textit{#1}}}}
\newcommand{\KeywordTok}[1]{\textcolor[rgb]{0.13,0.29,0.53}{\textbf{#1}}}
\newcommand{\NormalTok}[1]{#1}
\newcommand{\OperatorTok}[1]{\textcolor[rgb]{0.81,0.36,0.00}{\textbf{#1}}}
\newcommand{\OtherTok}[1]{\textcolor[rgb]{0.56,0.35,0.01}{#1}}
\newcommand{\PreprocessorTok}[1]{\textcolor[rgb]{0.56,0.35,0.01}{\textit{#1}}}
\newcommand{\RegionMarkerTok}[1]{#1}
\newcommand{\SpecialCharTok}[1]{\textcolor[rgb]{0.00,0.00,0.00}{#1}}
\newcommand{\SpecialStringTok}[1]{\textcolor[rgb]{0.31,0.60,0.02}{#1}}
\newcommand{\StringTok}[1]{\textcolor[rgb]{0.31,0.60,0.02}{#1}}
\newcommand{\VariableTok}[1]{\textcolor[rgb]{0.00,0.00,0.00}{#1}}
\newcommand{\VerbatimStringTok}[1]{\textcolor[rgb]{0.31,0.60,0.02}{#1}}
\newcommand{\WarningTok}[1]{\textcolor[rgb]{0.56,0.35,0.01}{\textbf{\textit{#1}}}}
\usepackage{graphicx,grffile}
\makeatletter
\def\maxwidth{\ifdim\Gin@nat@width>\linewidth\linewidth\else\Gin@nat@width\fi}
\def\maxheight{\ifdim\Gin@nat@height>\textheight\textheight\else\Gin@nat@height\fi}
\makeatother
% Scale images if necessary, so that they will not overflow the page
% margins by default, and it is still possible to overwrite the defaults
% using explicit options in \includegraphics[width, height, ...]{}
\setkeys{Gin}{width=\maxwidth,height=\maxheight,keepaspectratio}
% Set default figure placement to htbp
\makeatletter
\def\fps@figure{htbp}
\makeatother
\setlength{\emergencystretch}{3em} % prevent overfull lines
\providecommand{\tightlist}{%
  \setlength{\itemsep}{0pt}\setlength{\parskip}{0pt}}
\setcounter{secnumdepth}{-\maxdimen} % remove section numbering

\title{Introduction Air Pollution Project}
\author{R Programming}
\date{4/5/2020}

\begin{document}
\maketitle

\hypertarget{introduction}{%
\subsection{Introduction}\label{introduction}}

For this first programming assignment you will write three functions
that are meant to interact with dataset that accompanies this
assignment. The dataset is contained in a zip file \textbf{specdata.zip}
that you can download from the Coursera web site.

\textbf{Although this is a programming assignment, you will be assessed
using a separate quiz.}

\hypertarget{data}{%
\subsection{Data}\label{data}}

The zip file containing the data can be downloaded here:

\begin{itemize}
\tightlist
\item
  \href{https://d396qusza40orc.cloudfront.net/rprog\%2Fdata\%2Fspecdata.zip}{specdata.zip}
  {[}2.4MB{]}
\end{itemize}

The zip file contains 332 comma-separated-value (CSV) files containing
pollution monitoring data for fine particulate matter (PM) air pollution
at 332 locations in the United States. Each file contains data from a
single monitor and the ID number for each monitor is contained in the
file name. For example, data for monitor 200 is contained in the file
``200.csv''. Each file contains three variables:

\begin{itemize}
\tightlist
\item
  Date: the date of the observation in YYYY-MM-DD format
  (year-month-day)
\item
  sulfate: the level of sulfate PM in the air on that date (measured in
  micrograms per cubic meter)
\item
  nitrate: the level of nitrate PM in the air on that date (measured in
  micrograms per cubic meter)
\end{itemize}

For this programming assignment you will need to unzip this file and
create the directory `specdata'. Once you have unzipped the zip file, do
not make any modifications to the files in the `specdata' directory. In
each file you'll notice that there are many days where either sulfate or
nitrate (or both) are missing (coded as NA). This is common with air
pollution monitoring data in the United States.

\hypertarget{part-1}{%
\subsection{Part 1}\label{part-1}}

Write a function named `pollutantmean' that calculates the mean of a
pollutant (sulfate or nitrate) across a specified list of monitors. The
function `pollutantmean' takes three arguments: `directory',
`pollutant', and `id'. Given a vector monitor ID numbers,
`pollutantmean' reads that monitors' particulate matter data from the
directory specified in the `directory' argument and returns the mean of
the pollutant across all of the monitors, ignoring any missing values
coded as NA. A prototype of the function is as follows

\includegraphics{https://d3c33hcgiwev3.cloudfront.net/imageAssetProxy.v1/AniR5o00EeWk4wrqfRkIMQ_26d94fc4f878a8b60240f6fda6e17f6c_Screen-Shot-2015-11-17-at-9.03.29-AM.png?expiry=1586217600000\&hmac=BMZ3R3HotIprqN36DsE94hqtTQu96HJ0_FBqvgdjuPg}

You can see some example output from this function below. The function
that you write should be able to match this output. Please save your
code to a file named pollutantmean.R.

\begin{Shaded}
\begin{Highlighting}[]
\KeywordTok{print}\NormalTok{(R.version.string)}
\end{Highlighting}
\end{Shaded}

\begin{verbatim}
## [1] "R version 3.6.3 (2020-02-29)"
\end{verbatim}

\begin{Shaded}
\begin{Highlighting}[]
\CommentTok{#source("pollutantmean.R")}
\CommentTok{#pollutantmean("specdata", "sulfate", 1:10)}
\CommentTok{## [1] 4.064128}

\CommentTok{#pollutantmean("specdata", "nitrate", 70:72)}
\CommentTok{## [1] 1.706047}

\CommentTok{#pollutantmean("specdata", "nitrate", 23)}
\CommentTok{## [1] 1.280833}
\end{Highlighting}
\end{Shaded}

\hypertarget{part-2}{%
\subsection{Part 2}\label{part-2}}

Write a function that reads a directory full of files and reports the
number of completely observed cases in each data file. The function
should return a data frame where the first column is the name of the
file and the second column is the number of complete cases. A prototype
of this function follows

\includegraphics{https://d3c33hcgiwev3.cloudfront.net/imageAssetProxy.v1/Jnt5oY00EeWisRLkE7o57Q_2713e281672695ec59b29f83ec95f7b1_Screen-Shot-2015-11-17-at-9.04.23-AM.png?expiry=1586217600000\&hmac=uRWo8dDQxDhaijugcGME_kQ25y0CVF5uyFrm_iBn1FU}

You can see some example output from this function below. The function
that you write should be able to match this output. Please save your
code to a file named complete.R. To run the submit script for this part,
make sure your working directory has the file complete.R in it.

\begin{Shaded}
\begin{Highlighting}[]
\CommentTok{#source("complete.R")}
\CommentTok{#complete("specdata", 1)}
\CommentTok{##   id nobs}
\CommentTok{## 1  1  117}

\CommentTok{#complete("specdata", c(2, 4, 8, 10, 12))}
\CommentTok{##   id nobs}
\CommentTok{## 1  2 1041}
\CommentTok{## 2  4  474}
\CommentTok{## 3  8  192}
\CommentTok{## 4 10  148}
\CommentTok{## 5 12   96}

\CommentTok{#complete("specdata", 30:25)}
\CommentTok{##   id nobs}
\CommentTok{## 1 30  932}
\CommentTok{## 2 29  711}
\CommentTok{## 3 28  475}
\CommentTok{## 4 27  338}
\CommentTok{## 5 26  586}
\CommentTok{## 6 25  463}

\CommentTok{#complete("specdata", 3)}
\CommentTok{##   id nobs}
\CommentTok{## 1  3  243}
\end{Highlighting}
\end{Shaded}

\hypertarget{part-3}{%
\subsection{Part 3}\label{part-3}}

Write a function that takes a directory of data files and a threshold
for complete cases and calculates the correlation between sulfate and
nitrate for monitor locations where the number of completely observed
cases (on all variables) is greater than the threshold. The function
should return a vector of correlations for the monitors that meet the
threshold requirement. If no monitors meet the threshold requirement,
then the function should return a numeric vector of length 0. A
prototype of this function follows

\includegraphics{https://d3c33hcgiwev3.cloudfront.net/imageAssetProxy.v1/OXaiR400EeWk4wrqfRkIMQ_dafbb49ef127335cf1f9468fcadbd4ee_Screen-Shot-2015-11-17-at-9.05.01-AM.png?expiry=1586217600000\&hmac=bLudh6IeD3jtQV3B7X44G6Pv2ZUO7A7oBClbxeoCO_w}

For this function you will need to use the `cor' function in R which
calculates the correlation between two vectors. Please read the help
page for this function via `?cor' and make sure that you know how to use
it.

You can see some example output from this function below. The function
that you write should be able to approximately match this output.
\textbf{Note that because of how R rounds and presents floating point
numbers, the output you generate may differ slightly from the example
output.} Please save your code to a file named corr.R. To run the submit
script for this part, make sure your working directory has the file
corr.R in it.

\begin{Shaded}
\begin{Highlighting}[]
\KeywordTok{print}\NormalTok{(R.version.string)}
\end{Highlighting}
\end{Shaded}

\begin{verbatim}
## [1] "R version 3.6.3 (2020-02-29)"
\end{verbatim}

\begin{Shaded}
\begin{Highlighting}[]
\CommentTok{#source("corr.R")}
\CommentTok{#source("complete.R")}
\CommentTok{#cr <- corr("specdata", 150)}
\CommentTok{#head(cr)}
\CommentTok{## [1] -0.01895754 -0.14051254 -0.04389737 -0.06815956 -0.12350667 -0.07588814}

\CommentTok{#summary(cr)}
\CommentTok{##     Min.  1st Qu.   Median     Mean  3rd Qu.     Max. }
\CommentTok{## -0.21057 -0.04999  0.09463  0.12525  0.26844  0.76313}

\CommentTok{#cr <- corr("specdata", 400)}
\CommentTok{#head(cr)}
\CommentTok{## [1] -0.01895754 -0.04389737 -0.06815956 -0.07588814  0.76312884 -0.15782860}

\CommentTok{#summary(cr)}
\CommentTok{##     Min.  1st Qu.   Median     Mean  3rd Qu.     Max. }
\CommentTok{## -0.17623 -0.03109  0.10021  0.13969  0.26849  0.76313}

\CommentTok{#cr <- corr("specdata", 5000)}
\CommentTok{#summary(cr)}
\CommentTok{##    Min. 1st Qu.  Median    Mean 3rd Qu.    Max. }

\CommentTok{#length(cr)}
\CommentTok{## [1] 0}

\CommentTok{#cr <- corr("specdata")}
\CommentTok{#summary(cr)}
\CommentTok{##     Min.  1st Qu.   Median     Mean  3rd Qu.     Max. }
\CommentTok{## -1.00000 -0.05282  0.10718  0.13684  0.27831  1.00000}

\CommentTok{#length(cr)}
\CommentTok{## [1] 323}
\end{Highlighting}
\end{Shaded}

\end{document}
